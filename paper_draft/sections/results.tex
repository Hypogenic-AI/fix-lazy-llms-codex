\section{Results}
\label{sec:results}

\para{Main accuracy results.} Table~\ref{tab:accuracy} shows accuracy for all prompt conditions. \cotprompt is the strongest overall improvement, lifting \gsm from 0.52 to 0.74, while \rudedirect yields the highest \arc accuracy (0.92). Both harsh-critic variants reduce accuracy substantially on both datasets.

\begin{table}[t]
    \centering
    \resizebox{\textwidth}{!}{%
    \begin{tabular}{@{}lcc@{}}
        \toprule
        \textbf{Condition} & \textbf{\gsm Acc.} & \textbf{\arc Acc.} \\
        \midrule
        \direct & 0.52 & 0.88 \\
        \cotprompt & \textbf{0.74} & 0.90 \\
        \harshcritic & 0.40 & 0.32 \\
        \harshbudget & 0.14 & 0.24 \\
        \rudedirect & 0.48 & \textbf{0.92} \\
        \politedirect & 0.48 & 0.88 \\
        \bottomrule
    \end{tabular}
    }
    \caption{Accuracy on \gsm and \arc (n=50 each). Best results per dataset are in \textbf{bold}.}
    \label{tab:accuracy}
\end{table}


\para{Statistical comparisons.} Table~\ref{tab:diff_ci} reports paired bootstrap confidence intervals relative to \direct. \cotprompt provides a reliable gain on \gsm (+0.22, 95\% CI [0.08, 0.36]) and a small, non-significant change on \arc. In contrast, \harshcritic and \harshbudget show large negative deltas on both datasets, and the low-budget critic condition is the worst overall.

\begin{table}[t]
    \centering
    \resizebox{\textwidth}{!}{%
    \begin{tabular}{@{}lcc@{}}
        \toprule
        \textbf{Condition} & \textbf{\gsm Diff vs. \direct} & \textbf{\arc Diff vs. \direct} \\
        \midrule
        \cotprompt & +0.22 [0.08, 0.36] & +0.02 [-0.04, 0.08] \\
        \harshcritic & -0.12 [-0.26, 0.02] & -0.56 [-0.72, -0.40] \\
        \harshbudget & -0.38 [-0.52, -0.22] & -0.64 [-0.78, -0.50] \\
        \rudedirect & -0.04 [-0.12, 0.04] & +0.04 [0.00, 0.10] \\
        \politedirect & -0.04 [-0.12, 0.04] & 0.00 [0.00, 0.00] \\
        \bottomrule
    \end{tabular}
    }
    \caption{Paired bootstrap confidence intervals (95\%) for accuracy differences versus \direct.}
    \label{tab:diff_ci}
\end{table}


\para{Length-effort trade-offs.} Direct prompts are extremely short (about 2 words on average). \cotprompt responses are longer (roughly 103--123 words) and more accurate. \harshcritic produces the longest responses (roughly 155--191 words) while still reducing accuracy, and \harshbudget reduces length (about 85--99 words) but further harms accuracy. This pattern suggests that longer outputs are not inherently better when the critique signal is weak.

\para{Figures.} \Figref{fig:gsm8k_accuracy} and \Figref{fig:arc_accuracy} visualize the accuracy pattern across conditions, and \Figref{fig:question_length} shows the question length distribution for the evaluation samples.

\begin{figure}[t]
    \centering
    \includegraphics[width=0.95\linewidth]{figures/gsm8k_accuracy.png}
    \caption{Accuracy by condition on \gsm (n=50). \cotprompt improves accuracy, while harsh-critic variants reduce performance.}
    \label{fig:gsm8k_accuracy}
\end{figure}

\begin{figure}[t]
    \centering
    \includegraphics[width=0.95\linewidth]{figures/arc_accuracy.png}
    \caption{Accuracy by condition on \arc (n=50). Harsh-critic prompting sharply reduces accuracy despite longer outputs.}
    \label{fig:arc_accuracy}
\end{figure}

\begin{figure}[t]
    \centering
    \includegraphics[width=0.95\linewidth]{figures/question_length_hist.png}
    \caption{Question length distribution for the evaluation samples.}
    \label{fig:question_length}
\end{figure}
